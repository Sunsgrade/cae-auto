\documentclass[12pt,a4paper]{article}

% 中文支持(如不需要可删掉)
\usepackage[UTF8]{ctex}

% 数学公式与符号
\usepackage{amsmath,amssymb,amsfonts}

% 页面设置
\usepackage{geometry}
\geometry{a4paper, margin=1in}

% 代码与颜色
\usepackage{listings}
\usepackage{xcolor}

% 超链接
\usepackage{hyperref}

\title{欧拉梁振动推导示例}
\author{ChatGPT}
\date{\today}

\begin{document}

\maketitle

\tableofcontents
\newpage

\section{引言}
本文展示一个欧拉梁振动问题的推导示例。  
通过分离变量法求解微分方程,并给出特征方程。

\section{控制方程}
梁的弯曲振动微分方程为:
\begin{equation}
EI \frac{d^4 w(x,t)}{dx^4} + \rho A \frac{\partial^2 w(x,t)}{\partial t^2} = 0
\end{equation}

设
\[
w(x,t) = Q(x)\cos(\omega t)
\]
代入得:
\begin{equation}
EI Q^{(4)}(x) - \rho A \omega^2 Q(x) = 0
\end{equation}

\section{通解与边界条件}
通解为:
\[
Q(x) = A\sin(ax) + B\cos(ax) + C\sinh(ax) + D\cosh(ax)
\]

悬臂梁的边界条件为:
\[
\begin{cases}
Q(0) = 0, \\
Q'(0) = 0, \\
EI Q''(L) = 0, \\
EI Q'''(L) = -M\omega^2 Q(L)
\end{cases}
\]

\section{特征方程}
经过代入与整理,得到特征方程:
\begin{equation}
\cos(aL)\cosh(aL) + 1 = 0
\end{equation}

\section{数值解}
前几阶根为:
\[
aL = 1.875, \quad 4.694, \quad 7.855, \quad 10.996, \ldots
\]

\section{结论}
本文推导了悬臂梁的自由振动方程,并给出了特征值形式。

\end{document}